\section{Conclusiones}

Tras realizar el proyecto y revisar los objetivos, la impresión que me queda es que Moodle y Telegram pueden llegar a ser muy útiles trabajando juntos. Si bien con algún matiz, ya que Moodle me parece muy rígido en cuanto al modo de configurarlo para poder usar su API.
\par
El uso de Telegram facilita el contacto entre los estudiantes y los profesores de un curso, pudiendo llegar a ser el bot una herramienta que ayude, en gran medida, en todas aquellas tareas que se deriven de esta comunicación. Debido a la falta de tiempo hay muchas funcionalidades interesantes pensadas que no se han podido implementar como por ejemplo:
\begin{itemize}
\item Permitir imágenes en las dudas y las respuestas. 
\item Permitir a un profesor configurar funciones en el bot como:
\begin{itemize}
\item Mensaje bienvenida cuando un estudiante entra al chat del curso. 
\item Mostrar avisos cuanto falten x horas para que finalice la entrega de algún hito del curso.
\item Permitir programar que muestre un mensaje a las x horas del día y.
\end{itemize}
\item El bot notifique al estudiante cuando se ha aprobado/denegado su petición.
\item Permitir al bot \enquote*{grabar} la duda directamente del chat grupal, es decir, con un mecanismo tipo  \texttt{/duda ¿Cursiva o entrecomillado para los extranjerismos?} y grabar todas las respuestas que se dieran en el chat hasta pararlo con un  \texttt{/fin\_duda}.
\item El bot pudiera guardar recursos de interés compartidos por los estudiantes/profesores dentro del chat grupal.
\end{itemize}


Como última reflexión decir que Moodle no es imprescindible. La mayoría de las funcionalidades descritas más arriba no necesitan necesariamente a Moodle. 
Moodle aporta \textbf{control} e \textbf{información}, es el que aporta al bot quién es el responsable de qué curso y, por tanto, el responsable de indicarle al bot cuál es el chat en el que está \textbf{autorizado} a compartir la información del curso. Sin una fuente, el bot no sabe discernir dónde puede revelar esta o aquella información. Resulta muy interesante explorar otras posibles fuentes de información, como por ejemplo Google spreadsheets, Github o similares. 