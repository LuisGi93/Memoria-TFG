\chapter*{}
%\thispagestyle{empty}
%\cleardoublepage

%\thispagestyle{empty}

%\input{portada/portada_2}



\cleardoublepage
\thispagestyle{empty}

\begin{center}
{\large\bfseries Título del Proyecto: Subtítulo del proyecto}\\
\end{center}
\begin{center}
Nombre Apellido1 Apellido2 (alumno)\\
\end{center}

%\vspace{0.7cm}
\noindent{\textbf{Palabras clave}: Bot, Telegram, Moodle, Docencia, Chats, Mensajes ......}\\

\vspace{0.7cm}
\noindent{\textbf{Resumen}}\\

Hoy en día los bots están surgiendo para asistir en múltiples tareas y una plataforma que presta un fuerte apoyo a estos es Telegram. El objetivo del proyecto es utilizar un bot de Telegram y la plataforma Moodle para  facilitar las tareas relacionadas con la docencia a  profesores y alumnos.

\cleardoublepage


\thispagestyle{empty}


\begin{center}
{\large\bfseries Project Title: Project Subtitle}\\
\end{center}
\begin{center}
First name, Family name (student)\\
\end{center}

%\vspace{0.7cm}
\noindent{\textbf{Keywords}: Telegram, Bot, Moodle, Educative, Message, Chat}\\

\vspace{0.7cm}
\noindent{\textbf{Abstract}}\\


Nowadays bots are making their own way onto our lives. From the robot voice that guides you when you give the telephone company a call  to Youtube where bots supervise chats ensuring people don't employ bad words they can be used for a lot of useful tasks.

\par

Recently they have reached Telegram. Since October 2016 Telegram has released an API where programmers and enthusiasts can build their own bot for a lot of different purposes: greeting people who enter in a chat, looking for gifs, looking for music, travel plans, film recommendations... but in educative tasks, they are rarely used.

\par

The immediate nature of instant messages and the widespread use of programs like Whatssap and Telegram can captivate students and professors to  use them for educative purposes contributing in their way to reach their career goals. Believing in this idea the aim of this work is to build a Telegram bot that assists university courses in some of their day-to-day problems like helping solve doubts, providing students with the date of the upcoming delivers, getting their marks, managing professors academic tutoring... 
\par
So where are we going to find all the students and professor? We are going to use Moodle which is one of the most widely used educative platform in the university world.
\par
Moodle since version 2.0 provides an API that can be accessed using REST and enables external applications to interact with it so they can  make ordinary tasks like enrolling students, getting marks, adding resources to courses... Configuring Moodle correctly can be a tricky topic but knowing that security is a must we are going to try to develop a Moodle special configuration so the Moodle courses that use our bot don't disturb the others courses which don't.
\par 
We will use the special graphical keyboards and the menus that Telegram provides to bots to interact with users. With these special keyboards bots can easily indicate the user all the tasks enabled to them. We are also using one-step commandos in group chats instead of menus so they provide users with useful information causing the less noisy possible.
\par
We are going to employ collaborative technology like github and git as version control system so other people could contribute to help us develop our bot and hopefully making us useful suggestions that enable us to build a quality product. In addition we will employ technology like Vagrant to automatically build virtual machines and Ansible to provision this virtual machines to ease the problem of configuring and installing our bot. We will write unit tests becouse we want to make sure adding new features don't compromise the overall stability of the project.
\par

Finally we will use the Ruby programming language to develop the bot. Even though is not as popular as it used to be it has proven to have enough tools to ensure the construction of successful projects. Once it is finished and  because we believe in open source software we will release it under the MIT licence one of the less restrictive licences. 


\chapter*{}
\thispagestyle{empty}

\noindent\rule[-1ex]{\textwidth}{2pt}\\[4.5ex]

Yo, \textbf{Luis Gil Guijarro}, alumno de la titulación Grado en Ingeniería Informática de la \textbf{Escuela Técnica Superior
de Ingenierías Informática y de Telecomunicación de la Universidad de Granada}, con DNI 44066149-N, autorizo la
ubicación de la siguiente copia de mi Trabajo Fin de Grado en la biblioteca del centro para que pueda ser
consultada por las personas que lo deseen.

\vspace{6cm}

\noindent Fdo: Luis Gil Guijarro

\vspace{2cm}

\begin{flushright}
Granada a 1 de Septiembre de 2017 .
\end{flushright}


\chapter*{}
\thispagestyle{empty}

\noindent\rule[-1ex]{\textwidth}{2pt}\\[4.5ex]

D. \textbf{Nombre Apellido1 Apellido2 (tutor1)}, Profesor del Área de XXXX del Departamento YYYY de la Universidad de Granada.

\vspace{0.5cm}

D. \textbf{Nombre Apellido1 Apellido2 (tutor2)}, Profesor del Área de XXXX del Departamento YYYY de la Universidad de Granada.


\vspace{0.5cm}

\textbf{Informan:}

\vspace{0.5cm}

Que el presente trabajo, titulado \textit{\textbf{Título del proyecto, Subtítulo del proyecto}},
ha sido realizado bajo su supervisión por \textbf{Nombre Apellido1 Apellido2 (alumno)}, y autorizamos la defensa de dicho trabajo ante el tribunal
que corresponda.

\vspace{0.5cm}

Y para que conste, expiden y firman el presente informe en Granada a X de mes de 201 .

\vspace{1cm}

\textbf{Los directores:}

\vspace{5cm}

\noindent \textbf{Nombre Apellido1 Apellido2 (tutor1) \ \ \ \ \ Nombre Apellido1 Apellido2 (tutor2)}

\chapter*{Agradecimientos}
\thispagestyle{empty}

       \vspace{1cm}


Poner aquí agradecimientos...
También a todo profesor que ha desmanejado mis dudas en su despacho y a aquellos que no me han bajado la nota al ir después del examen.

