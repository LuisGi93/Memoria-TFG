\documentclass[a4paper,11pt]{book}
%\documentclass[a4paper,twoside,11pt,titlepage]{book}
\usepackage{listings}
\usepackage[utf8]{inputenc}
\usepackage[spanish]{babel}

% \usepackage[style=list, number=none]{glossary} %
%\usepackage{titlesec}
%\usepackage{pailatino}

\decimalpoint
\usepackage{dcolumn}
\newcolumntype{.}{D{.}{\esperiod}{-1}}
\makeatletter
\addto\shorthandsspanish{\let\esperiod\es@period@code}
\makeatother


%\usepackage[chapter]{algorithm}
\RequirePackage{verbatim}
%\RequirePackage[Glenn]{fncychap}
\usepackage{fancyhdr}
\usepackage{graphicx}
\usepackage{afterpage}


\usepackage[export]{adjustbox}

\usepackage{longtable}

\usepackage[pdfborder={000}]{hyperref} %referencia

% ********************************************************************
% Re-usable information
% ********************************************************************
\newcommand{\myTitle}{Título del proyecto\xspace}
\newcommand{\myDegree}{Grado en ...\xspace}
\newcommand{\myName}{Nombre Apllido1 Apellido2 (alumno)\xspace}
\newcommand{\myProf}{Nombre Apllido1 Apellido2 (tutor1)\xspace}
\newcommand{\myOtherProf}{Nombre Apllido1 Apellido2 (tutor2)\xspace}
%\newcommand{\mySupervisor}{Put name here\xspace}
\newcommand{\myFaculty}{Escuela Técnica Superior de Ingenierías Informática y de
Telecomunicación\xspace}
\newcommand{\myFacultyShort}{E.T.S. de Ingenierías Informática y de
Telecomunicación\xspace}
\newcommand{\myDepartment}{Departamento de ...\xspace}
\newcommand{\myUni}{\protect{Universidad de Granada}\xspace}
\newcommand{\myLocation}{Granada\xspace}
\newcommand{\myTime}{\today\xspace}
\newcommand{\myVersion}{Version 0.1\xspace}


\hypersetup{
pdfauthor = {\myName (email (en) ugr (punto) es)},
pdftitle = {\myTitle},
pdfsubject = {},
pdfkeywords = {palabra_clave1, palabra_clave2, palabra_clave3, ...},
pdfcreator = {LaTeX con el paquete ....},
pdfproducer = {pdflatex}
}

%\hyphenation{}


%\usepackage{doxygen/doxygen}
%\usepackage{pdfpages}
\usepackage{url}
\usepackage{colortbl,longtable}
\usepackage[stable]{footmisc}
%\usepackage{index}

%\makeindex
%\usepackage[style=long, cols=2,border=plain,toc=true,number=none]{glossary}
% \makeglossary

% Definición de comandos que me son tiles:
%\renewcommand{\indexname}{Índice alfabético}
%\renewcommand{\glossaryname}{Glosario}

\pagestyle{fancy}
\fancyhf{}
\fancyhead[LO]{\leftmark}
\fancyhead[RE]{\rightmark}
\fancyhead[RO,LE]{\textbf{\thepage}}
\renewcommand{\chaptermark}[1]{\markboth{\textbf{#1}}{}}
\renewcommand{\sectionmark}[1]{\markright{\textbf{\thesection. #1}}}

\setlength{\headheight}{1.5\headheight}

\newcommand{\HRule}{\rule{\linewidth}{0.5mm}}
%Definimos los tipos teorema, ejemplo y definición podremos usar estos tipos
%simplemente poniendo \begin{teorema} \end{teorema} ...
\newtheorem{teorema}{Teorema}[chapter]
\newtheorem{ejemplo}{Ejemplo}[chapter]
\newtheorem{definicion}{Definición}[chapter]

\definecolor{gray97}{gray}{.97}
\definecolor{gray75}{gray}{.75}
\definecolor{gray45}{gray}{.45}
\definecolor{gray30}{gray}{.94}

\lstset{ frame=Ltb,
     framerule=0.5pt,
     aboveskip=0.5cm,
     framextopmargin=3pt,
     framexbottommargin=3pt,
     framexleftmargin=0.1cm,
     framesep=0pt,
     rulesep=.4pt,
     backgroundcolor=\color{gray97},
     rulesepcolor=\color{black},
     %
     stringstyle=\ttfamily,
     showstringspaces = false,
     basicstyle=\scriptsize\ttfamily,
     commentstyle=\color{gray45},
     keywordstyle=\bfseries,
     %
     numbers=left,
     numbersep=6pt,
     numberstyle=\tiny,
     numberfirstline = false,
     breaklines=true,
   }
 
% minimizar fragmentado de listados
\lstnewenvironment{listing}[1][]
   {\lstset{#1}\pagebreak[0]}{\pagebreak[0]}

\lstdefinestyle{CodigoC}
   {
	basicstyle=\scriptsize,
	frame=single,
	language=C,
	numbers=left
   }
\lstdefinestyle{CodigoC++}
   {
	basicstyle=\small,
	frame=single,
	backgroundcolor=\color{gray30},
	language=C++,
	numbers=left
   }

 
\lstdefinestyle{Consola}
   {basicstyle=\scriptsize\bf\ttfamily,
    backgroundcolor=\color{gray30},
    frame=single,
    numbers=none
   }


\newcommand{\bigrule}{\titlerule[0.5mm]}


%Para conseguir que en las páginas en blanco no ponga cabecerass
\makeatletter
\def\clearpage{%
  \ifvmode
    \ifnum \@dbltopnum =\m@ne
      \ifdim \pagetotal <\topskip
        \hbox{}
      \fi
    \fi
  \fi
  \newpage
  \thispagestyle{empty}
  \write\m@ne{}
  \vbox{}
  \penalty -\@Mi
}
\makeatother

\usepackage{pdfpages}


%Paquetes añadidos por mi
\newcommand{\tabitem}{~~\llap{\textbullet}~~}

\setcounter{secnumdepth}{5} %subsections number deep 5

\usepackage{lipsum} % dummy text for the MWE

% Put % before of what you want disabled

% Select what to do with todonotes: 
% \usepackage[disable]{todonotes} % notes not showed
\usepackage[draft]{todonotes}   % notes showed

%para que no floten las tablas
\usepackage{float}
\restylefloat{table}

\usepackage[square,numbers,sectionbib]{natbib}
\usepackage{chapterbib}

\usepackage{csquotes}

%%Colores tablas
\usepackage{color, colortbl}
\definecolor{LightCyan}{rgb}{0.88,1,1}
\definecolor{Gray}{gray}{0.9}



%mostrar codigo



\usepackage{listings}
\usepackage{color}

\definecolor{dkgreen}{rgb}{0,0.6,0}
\definecolor{gray}{rgb}{0.5,0.5,0.5}
\definecolor{mauve}{rgb}{0.58,0,0.82}

\lstset{frame=tb,
  language=Java,
  aboveskip=3mm,
  belowskip=3mm,
  showstringspaces=false,
  columns=flexible,
  basicstyle={\small\ttfamily},
  numbers=none,
  numberstyle=\tiny\color{gray},
  keywordstyle=\color{blue},
  commentstyle=\color{dkgreen},
  stringstyle=\color{mauve},
  breaklines=true,
  breakatwhitespace=true,
  tabsize=3
}


%% ajustar figura planificacion
\usepackage[export]{adjustbox}

\usepackage{eurosym}



%%%%


\lstloadlanguages{Ruby}
\lstset{%
basicstyle=\ttfamily\color{black},
commentstyle = \ttfamily\color{red},
keywordstyle=\ttfamily\color{blue},
stringstyle=\color{orange}}


%fin paquetes añadidos por mi


\begin{document}
\input{portada/portada}
\chapter*{}
%\thispagestyle{empty}
%\cleardoublepage

%\thispagestyle{empty}

%\input{portada/portada_2}



\cleardoublepage
\thispagestyle{empty}

\begin{center}
{\large\bfseries Pradobot: \textit{Bot} de Telegram para Moodle}\\
\end{center}
\begin{center}
Luis Gil Guijarro\\
\end{center}

%\vspace{0.7cm}
\noindent{\textbf{Palabras clave}: \textit{Bot}, Telegram, Moodle, Docencia, Chat, Mensaje}\\

\vspace{0.7cm}
\noindent{\textbf{Resumen}}\\

Hoy en día los \textit{bots} están surgiendo para asistir en múltiples tareas y una plataforma que presta un fuerte apoyo a estos es Telegram. El objetivo del proyecto es utilizar un \textit{bot} de Telegram y la plataforma Moodle para  facilitar las tareas relacionadas con la docencia a  profesores y alumnos.

\cleardoublepage


\thispagestyle{empty}


\begin{center}
{\large\bfseries Pradobot: Telegram bot for Moodle}\\
\end{center}
\begin{center}
Luis Gil Guijarro\\
\end{center}

%\vspace{0.7cm}
\noindent{\textbf{Keywords}: Telegram, Bot, Moodle, Educative, Message, Chat}\\

\vspace{0.7cm}
\noindent{\textbf{Abstract}}\\


Nowadays bots are making their own way onto our lives. From the robot voice that guides you when you give the telephone company a call  to Youtube where bots supervise chats ensuring people don't employ bad words they can be used for a lot of useful tasks.

\par

Recently they have reached Telegram. Since October 2016 Telegram has released an API where programmers and enthusiasts can build their own bot for a lot of different purposes: greeting people who enter in a chat, looking for gifs, looking for music, travel plans, film recommendations... but in educative tasks, they are rarely used.

\par

The immediate nature of instant messages and the widespread use of programs like Whatssap and Telegram can captivate students and professors to  use them for educative purposes contributing in their way to reach their career goals. Believing in this idea the aim of this work is to build a Telegram bot that assists university courses in some of their day-to-day problems like helping solve doubts, providing students with the date of the upcoming delivers, getting their marks, managing professors academic tutoring... 
\par
So where are we going to find all the students and professor? We are going to use Moodle which is one of the most widely used educative platform in the university world.
\par
Moodle since version 2.0 provides an API that can be accessed using REST and enables external applications to interact with it so they can  make ordinary tasks like enrolling students, getting marks, adding resources to courses... Configuring Moodle correctly can be a tricky topic but knowing that security is a must we are going to try to develop a Moodle special configuration so the Moodle courses that use our bot don't disturb the others courses which don't.
\par 
We will use the special graphical keyboards and the menus that Telegram provides to bots to interact with users. With these special keyboards bots can easily indicate the user all the tasks enabled to them. We are also using one-step commandos in group chats instead of menus so they provide users with useful information causing the less noisy possible.
\par
We are going to employ collaborative technology like github and git as version control system so other people could contribute to help us develop our bot and hopefully making us useful suggestions that enable us to build a quality product. In addition we will employ technology like Vagrant to automatically build virtual machines and Ansible to provision this virtual machines to ease the problem of configuring and installing our bot. We will write unit tests becouse we want to make sure adding new features don't compromise the overall stability of the project.
\par

Finally we will use the Ruby programming language to develop the bot. Even though is not as popular as it used to be it has proven to have enough tools to ensure the construction of successful projects. Once it is finished and  because we believe in open source software we will release it under the MIT licence one of the less restrictive licences. 


\chapter*{}
\thispagestyle{empty}

\noindent\rule[-1ex]{\textwidth}{2pt}\\[4.5ex]

Yo, \textbf{Luis Gil Guijarro}, alumno de la titulación Grado en Ingeniería Informática de la \textbf{Escuela Técnica Superior
de Ingenierías Informática y de Telecomunicación de la Universidad de Granada}, con DNI 44066149-N, autorizo la
ubicación de la siguiente copia de mi Trabajo Fin de Grado en la biblioteca del centro para que pueda ser
consultada por las personas que lo deseen.

\vspace{6cm}

\noindent Fdo: Luis Gil Guijarro

\vspace{2cm}

\begin{flushright}
Granada a 10 de Septiembre de 2017 .
\end{flushright}


\chapter*{}
\thispagestyle{empty}

\noindent\rule[-1ex]{\textwidth}{2pt}\\[4.5ex]

D. \textbf{Juan Julián Merelo Guervós}, profesor del \textbf{Departamento de Arquitectura y Tecnología de Computadores} de la \textbf{Universidad de Granada}.


\vspace{0.5cm}

\textbf{Informa:}

\vspace{0.5cm}

Que el presente trabajo, titulado \textit{\textbf{Pradobot: Bot de Telegram para Moodle}},
ha sido realizado bajo su supervisión por \textbf{Luis Gil Guijarro}, y autorizamos la defensa de dicho trabajo ante el tribunal
que corresponda.

\vspace{0.5cm}

Y para que conste, expiden y firman el presente informe en Granada a 10 de septiembre de 2017 .

\vspace{1cm}

\textbf{El tutor:}

\vspace{5cm}

\noindent \textbf{Juan Julián Merelo Guervós}

\chapter*{Agradecimientos}
\thispagestyle{empty}

       \vspace{1cm}



Van a ser un poco largos porque por suerte ha habido bastantes personas que han influido en mi camino que culmina con este Trabajo Fin de Grado:\par 
\medskip

A mi familia, por darme la oportunidad de estudiar una carrera y estar ahí tanto para lo bueno como para lo malo.\par
\medskip

A mis profesores, por aclarar mis dudas y por su dedicación.\par
\medskip

A los dos ceros que me pusieron nada más llegar, porque sin ellos no me habría cabreado lo suficiente como para llegar a este Trabajo Fin de Grado.\par
\medskip

A los compañeros que arrimaron su hombro ante un trabajo complicado, porque sufrimiento compartido es mitad y al final sale.\par
\medskip

A JJ, porque habría salido de la carrera sin conocer muchas herramientas útiles y por ser mi tutor.\par
\medskip

A la ETSIIT, por usar Linux.\par
\medskip

Al profesor Antonio Troncoso Reigada, por preguntarme si me gustaba el derecho.\par


\frontmatter
\tableofcontents
\listoffigures
\listoftables
%
\mainmatter
\setlength{\parskip}{5pt}

\input{capitulos/01_Introduccion}
%
\input{capitulos/02_Planificacion}
%
\input{capitulos/03_EspecificacionRequisitos}
%
\input{capitulos/04_Analisis}
%
\input{capitulos/05_Diseno}
%
\input{capitulos/06_Implementacion}
%
\chapter{}

\section{Pruebas}

\subsection{Prueba despliegue}

Para probar la creación de la máquina virtual en Amazon AWS vamos a mostrar el resultados de ejecutar el Vagrantfile mostrado anteriormente. Es necesario ejecutar la orden \texttt{vagrant up --provider=aws}:
\begin{figure}[H] %con el [H] le obligamos a situar aquí la figura
\centering
\includegraphics[scale=0.3]{imagenes/random/2017-09-05-172749_1366x768_scrot.png}  %el parámetro scale permite agrandar o achicar la imagen. En el nombre de archivo puede especificar directorios

\caption{Ejecutando los tests de la aplicación}\label{figura92}

\end{figure}

En la salida de la orden se pueden ver las características de la máquina virtual creada tras lo cual una vez está habilitado ssh se ejecuta el playbook de Ansible:

\begin{figure}[H] %con el [H] le obligamos a situar aquí la figura
\centering
\includegraphics[scale=0.3]{imagenes/random/2017-09-05-174148_1366x768_scrot.png}  %el parámetro scale permite agrandar o achicar la imagen. En el nombre de archivo puede especificar directorios

\caption{Finalización de aprovisionamiento con Ansible}\label{figura94}

\end{figure}

Por último ejecutamos Capistrano para iniciar y parar la aplicación:

\begin{figure}[H] %con el [H] le obligamos a situar aquí la figura
\centering
\includegraphics[scale=0.3]{imagenes/random/2017-09-05-184638_1366x768_scrot.png}  %el parámetro scale permite agrandar o achicar la imagen. En el nombre de archivo puede especificar directorios

\caption{Iniciamos la ejecución de la aplicación utilizando Capistrano}\label{figura905}

\end{figure}

\begin{figure}[H] %con el [H] le obligamos a situar aquí la figura
\centering
\includegraphics[scale=0.3]{imagenes/random/2017-09-05-184725_1366x768_scrot.png}  %el parámetro scale permite agrandar o achicar la imagen. En el nombre de archivo puede especificar directorios

\caption{Detenemos la ejecución de la aplicación}\label{figura910}

\end{figure}


Capistrano puede que al hacer el \texttt{cap production pradobot:daemon\_start} se quede paralizado siendo necesario hacer un control-C para que se salga de la terminal. La aplicación se seguirá ejecutando normalmente.


\subsection{Tests}

Para ejecutar los tests unitarios que componen la aplicación es necesario ejecutar la orden \texttt{rake} en el directorio en el cual se encuentra el Rakefile. Esto hará que rake lea los ficheros especificados en el Rakefile y ejecute los tests contenidos en ellos uno a uno:

\begin{figure}[H] %con el [H] le obligamos a situar aquí la figura
\centering
\includegraphics[scale=0.4]{imagenes/random/Screenshot_2017-09-04_18-28-17.png}  %el parámetro scale permite agrandar o achicar la imagen. En el nombre de archivo puede especificar directorios

\caption{Tras ejecutar los tests de la aplicación}\label{figura92}

\end{figure}

Como podemos ver en la imagen superior se muestra en verde cada test que ha pasado satisfactoriamente dando al final un resumen del número de tests ejecutados (156 examples) y de cuántos han fallado (0 failures) indicando además cuánto tiempo se han tardado en ejecutar los tests. También se muestra el porcentaje de código que cubren nuestros tests (93.13\%) generándose un reporte en formato html que permite ver qué lineas de código no se han cubierto:

\begin{figure}[H] %con el [H] le obligamos a situar aquí la figura
\centering
\includegraphics[scale=0.3]{imagenes/random/Screenshot_2017-09-05_14-47-08.png}  %el parámetro scale permite agrandar o achicar la imagen. En el nombre de archivo puede especificar directorios

\caption{Reporte de cobertura de código de lo tests}\label{figura94}

\end{figure}

\subsection{Rendimiento}
Realizar test de rendimiento sobre el programa es complejo, ya que cualquier mensaje que se le mande al bot implicará llamadas a la base de datos y hacer mocks y stubs solamente es práctico si se va ha realizar un test sobre una parte del sistema. Aún así vamos a probar a mandar varios mensajes desde chats privados y grupales midiendo el tiempo que se tarda en procesar los mensajes. El código utilizado en la medición es el siguiente:
\begin{figure}[H] %con el [H] le obligamos a situar aquí la figura
\centering
\includegraphics[scale=0.6]{imagenes/random/rend1.png}  %el parámetro scale permite agrandar o achicar la imagen. En el nombre de archivo puede especificar directorios
\caption{Código utilizado para la medición del tiempo que se tarda en procesar un mensaje}\label{figura94}
\end{figure}



Se ha utilizado la función \texttt{measure} del módulo de benchmarking de ruby \href{https://ruby-doc.org/stdlib-2.0.0/libdoc/benchmark/rdoc/Benchmark.html}{Benchmark}. Esta imprime el tiempo de CPU en espacio de usuario (\textit{ CPU time}), tiempo en el kernel realizando operaciones para nuestro programa (\textit{system CPU time}), la suma de los dos  y después el tiempo que se ha empleado en total (\textit{elapsed real time}). Los tiempos obtenidos son los siguientes:

\begin{figure}[H] %con el [H] le obligamos a situar aquí la figura
\centering
\includegraphics[scale=0.5]{imagenes/random/rend2.png}  %el parámetro scale permite agrandar o achicar la imagen. En el nombre de archivo puede especificar directorios
\caption{Tiempo que tarda pradobot en procesar mensajes procedentes de un chat privado y de unchat grupal}\label{figura94}
\end{figure}

Como podemos observar los tiempos de CPU suelen rondar los 0.1 segundos, mientras que el tiempo real empleado es de entre 0.8-2 segundos. Esta diferencia es bastante significativa. Vamos a probar a mandar mensajes con otro programa  \enquote*{telegram-cli}. Este programa permite mandar mensajes a un chat de Telegram desde la terminal:


\begin{figure}[H] %con el [H] le obligamos a situar aquí la figura
\centering
\includegraphics[scale=0.4]{imagenes/random/Screenshot_2017-09-01_11-21-38}  %el parámetro scale permite agrandar o achicar la imagen. En el nombre de archivo puede especificar directorios
\caption{Envío de mensaje utilizando telegram-cli}\label{figura94}
\end{figure}

Se puede apreciar que la diferencia de tiempo entre el tiempo de CPU total y el tiempo real sigue siendo significativa. El porqué de esta diferencia pienso que es debido a la latencia que supone recibir y enviar un mensaje a los servidores de Telegram, es decir, en el procesamiento del mensaje llega un punto en que se inicia el envío del mensaje de respuesta a Telegram, el sistema operativo, mientras los datos transitan por la red, pausa el programa transfiriéndole el control de la CPU a otro proceso, se completa el envío y nuestro programa finaliza con el procesamiento del mensaje.
Lo cual implica que la velocidad con la que el usuario ve una respuesta por parte del bot no es tanto la tardanza de procesar el mensaje sino de la rapidez con la que se envían y reciben los mensajes pudiendo decirse, por tanto, que si quisiéramos que nuestro bot respondiera rápidamente no tendríamos que comprar un procesador más potente sino una conexión a internet más veloz.

%
\input{capitulos/08_Conclusiones}
%
%\input{capitulos/A0_er_normalizacion_bd}
%
\chapter{}
%
%
\section{Glosario de términos}

\textbf{Bot}: Programa informático que efectua tareas repetitivas a través de internet\cite{wikibot}\cite{wikibot2}.\par
\textbf{Telegram}: Servicio de mensajería instantanea por internet.\par
\textbf{Moodle}: Plataforma educativa diseñada para la gestión de cursos online.\par
\textbf{Interfaz}:Capa intermedia entre dos sistemas que utilizan para comunicarse.\par
\textbf{Desarrollo iterativo}: Metodología de desarrollo de software que implica agrupar el desarrollo del sofware en un conjunto de etapas repetitivas.\cite{wikiiterativo}, \cite{wikiiterativo2} \par
\textbf{SSH}: Protocolo para la comunicación segura entre computadores\cite{ssh}\par
\textbf{SSL}:Protocolo para la transmisión de manera segura de información entre dos aplicaciones\par
\textbf{Token}: Identificador que se le da a un usuario para facilitar el proceso de autenticación.\par
\textbf{API}:  interfaz de programación de aplicaciones, conjunto de funciones que ofrece una biblioteca o aplicación con el fin de que sea usado por otro software como una capa de abstracción.\par
\textbf{REST}:  Transferencia de Estado Representacional, estilo de arquitectura de software para sistemas distribuidos. Actualmente el término se usa para describir cualquier interfaz  que permita la modificación de la forma textual de recursos web a través de un conjunto de operaciones predefinidas y sin estado.\cite{rest1}\cite{rest2}\par
\textbf{pradobot}:  Nombre del programa que estamos desarrollando.\par
\textbf{Polimorfismo}: Capacidad de enviar el mismo tipo de mensajes a objetos de diferente clase. \par
\textbf{ORM}:  Es una técnica que permite realizar consultas y manipular datos de la base de datos realizando operaciones sobre  \enquote*{objetos} que simbolizan la estructura de los datos.\cite{orm1}\par
\textbf{Issue}:  Un issue de Github es como una llamada de atención sobre un repositorio de github.\par
\textbf{Repositorio}:De forma muy simple se puede describir Github como una página web que permite el desarrollo colaborativo de software siendo el repositorio de un proyecto el lugar donde se almacena su código.\par
\textbf{Stub}: Simulan un objeto con una serie de métodos que devuelven datos predefinidos..\cite{stub1}\par
\textbf{Mock}: Parecido a los stubs solo que sus métodos no tiene un comportamiento predefinido. Tienen programados una serie de expectativas como que se llame x veces a este método con tales parámetros y se suelen utilizar para si el uso de la clase que simulan es el esperado en los tests.\par
%\begin{thebibliography}{999}


%\bibliographystylePS{plain}
%\bibliographyPS{/bibliografia/moodle}

\begin{thebibliography}{99}

\subsubsection*{Ruby}
\bibitem{Ruby} ``Ruby docs''. 20/03/2017.\url{ruby-doc.org/}
\subsubsection*{Moodle}

\bibitem{mood1} ``Get users token''. 05/03/2017.\url{https://moodle.org/mod/forum/discuss.php?d=193857}
\bibitem{mood2} ``How to get a user token from a external aplciation''.05/03/2017. \url{https://docs.moodle.org/31/en/Web_services_FAQ#How_to_get_a_user_token_from_an_external_application.3F
}
\bibitem{mood3} ``Creating a web service client''. 06/03/2017.\url{https://docs.moodle.org/dev/Creating_a_web_service_client#How_to_get_a_user_token}
\bibitem{mood4} ``Add new user''. \url{https://docs.moodle.org/32/en/Add_a_new_user}
\bibitem{mood5} ``Step by step installation guide moodle ubuntu''. 07/03/2017.\url{https://docs.moodle.org/26/en/Step-by-step_Installation_Guide_for_Ubuntu
}
\bibitem{mood6} ``Roles and capabilities''. 07/03/2017.\url{https://docs.moodle.org/19/en/Roles_and_capabilities}

\bibitem{mood7} ``Webservice API functions''. 06/03/2017. \url{https://docs.moodle.org/dev/Web_service_API_functions}
\bibitem{mood8} ``Webservices''. 06/03/2017. \url{https://docs.moodle.org/dev/Web_services}

\subsubsection*{UML}
\bibitem{uml1} ``UML basics: The sequence diagram''. 09/04/2017. \url{https://www.ibm.com/developerworks/rational/library/3101.html}
\bibitem{uml2} ``Differences between sequence diagrams and collaboration diagrams''. 09/04/2017. \url{https://www-01.ibm.com/support/docview.wss?uid=swg21123475
}
\bibitem{uml2} ``UML 2 Sequence Diagrams: An Agile Introduction''. 09/04/2017. \url{http://www.agilemodeling.com/artifacts/sequenceDiagram.htm
}
\subsubsection*{Lecturas sobre bots}
\bibitem{abo1} ``Server-side architecture when bots invade''. 01/03/2017. \url{https://medium.com/@JonathanZWhite/server-side-infrastructure-when-bots-invade-a2252e9d4bc9}
\bibitem{abo2} ``Server-side architecture when bots invade''. 01/03/2017. \url{https://medium.com/@surmenok/chatbot-architecture-496f5bf820ed
}


\subsubsection*{Sequel}

\bibitem{sequel1} ``Dataset filtering''. 01/05/2017 . \url{http://sequel.jeremyevans.net/rdoc/files/doc/dataset_filtering_rdoc.html}
\bibitem{sequel2} ``ConnectionPool''. 01/05/2017 . \url{http://www.rubydoc.info/github/evanfarrar/opensprints/Sequel/ConnectionPool}
\bibitem{sequel3} ``ThreadedConnectionPool''. 01/05/2017 .\url{http://sequel.jeremyevans.net/rdoc/classes/Sequel/ThreadedConnectionPool.html#attribute-i-available_connections
}
\subsubsection*{Tests}

\bibitem{rspec} `` RSpec Expectations 3.6''. 10/05/2017 . \url{https://relishapp.com/rspec/rspec-expectations/docs/built-in-matchers}
\bibitem{rspec} `` RSpec Expectations 3.6''. 10/05/2017 . \url{https://dzone.com/articles/why-shouldnt-i-test-privates}

\subsubsection*{Telegram}
\bibitem{bot1} ``Telegram''. 01/03/2017 .\url{https://core.telegram.org/bots}

\bibitem{bot2} ``Telegram bot API''. 01/03/2017 . \url{https://core.telegram.org/bots/api#message}
\bibitem{bot3} ``Telegram bot updates''. 01/03/2017 . \url{https://core.telegram.org/bots/api#getupdates}
\bibitem{bot3} ``Long polling vs webhooks''. 01/03/2017 . \url{https://github.com/python-telegram-bot/python-telegram-bot/wiki/Webhooks#polling-vs-webhook}

\subsubsection*{Definiciones}
\bibitem{orm1} ``Visual paradigm ORM''. 01/09/2017 . \url{https://www.visual-paradigm.com/support/documents/vpuserguide/3563/3581/85424_whatisobject.html}
\bibitem{ssh} ``SSH''. 01/09/2017.\url{http://web.mit.edu/rhel-doc/4/RH-DOCS/rhel-rg-es-4/ch-ssh.html}
\bibitem{wikibot} ``Bot''. 01/09/2017 . \url{https://es.wikipedia.org/wiki/Bot}
\bibitem{wikibot2} ``Malicious bots''.01/09/2017. \url{https://books.google.com/books?id=nmgK7KcibSUC}
\bibitem{wikiiterativo} ``Desarrollo iterativo y creciente''.01/09/2017. \url{https://es.wikipedia.org/wiki/Desarrollo_iterativo_y_creciente}
\bibitem{wikiiterativo2} ``Proceso de Desarrollo Iterativo''. 01/09/2017 . \url{http://fernandosoriano.com.ar/?p=13}
\bibitem{rest1} ``REST''. 01/09/2017 . \url{https://www.w3.org/TR/2004/NOTE-ws-arch-20040211/#relwwwrest}
\bibitem{rest2} ``Representational state transfer''. 01/09/2017 . \url{https://en.wikipedia.org/wiki/Representational_state_transfer}
\bibitem{stub1} ``Pruebas unitarias''. 01/09/2017 . \url{https://itblogsogeti.com/2015/03/26/desarrollo-pruebas-unitarias-trinitario-gomez-sogeti/
}

\bigskip
\subsubsection*{Otro material}
\begin{itemize}
	\item Consultas {\tt Stack OverFlow}.
	\item Alguna cosilla latex {\tt https://github.com/germaaan/TFG}.
	\item Material de las asignaturas \textbf{Fundamentos de Ingeniería del Software}, \textbf{Programación Orientada a Objetos}, \textbf{Diseño de Interfaces de Usuario  }, \textbf{Diseño de Interfaces de Usuario  }, \textbf{Diseño y Desarrollo de Sistemas de Información}, \textbf{Diseño y Desarrollo de Sistemas de Información}, \textbf{Seguridad en Sistemas Operativos}, \textbf{Derecho Informático},  ( \textbf{Ingeniería de Servidores} e \textbf{Infraestructura Virtual} impartidas en \textbf{Grado en Ingeniería Informática} en la \textbf{Universidad de Granada}.
\end{itemize}
\end{thebibliography}
\addcontentsline{toc}{chapter}{Bibliografía}


%\bibliography{bibliografia/bibliografia}
%\bibliographystyle{plain}
%\bibliographystyle{miunsrturl}
%
%\appendix
%\input{apendices/manual_usuario/manual_usuario}
%%\input{apendices/paper/paper}
%\input{glosario/entradas_glosario}
% \addcontentsline{toc}{chapter}{Glosario}
% \printglossary
\chapter*{}

\thispagestyle{empty}

\end{document}
